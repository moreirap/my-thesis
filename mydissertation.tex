\documentclass[dissertation]{softeng}

\usepackage{times}

\title{Extending BDD\\A systematic approach to handling non-functional requirements}
\author{Pedro Moreira}
\college{Kellogg College}
\organisation{University of Oxford}
\award{Software Engineering}

\begin{document}

\maketitle


\begin{abstract}
Software engineering methods have evolved from having a prescribed and sequential nature to using more adaptable and iterative approaches. Such is the case with Behaviour Driven Development (BDD)~\cite{North2006,Smart201410}, a recent member of the family of agile methodologies addressing the correct specification of the behaviour characteristics of a system, by focusing on close collaboration and identification of examples.

Whilst BDD is very successful in ensuring developed software meets its functional requirements, it is largely silent regarding the systematic treatment of its non-functional counterparts~--- descriptions of how the system should behave with respect to some quality attribute such as performance, reusability, etc.

Historically, the systematic treatment of non-functional requirements (NFRs) in software engineering is categorised as being either product-oriented based on a quantitative approach aimed at evaluating the degree to which a system meets its non-functional requirements or process-oriented, qualitative in nature and where NFRs are used to drive the software design process.

One example of the latter category, is the NFR Framework~\cite{Chung2000}, a structured approach to represent and reason about non-functional requirements. 

In this thesis, we investigate the extent to which BDD can be integrated with the NFR Framework, with the aim of handling non-functional requirements in an explicit and systematic way whilst respecting the principles and philosophy behind Agile.
\end{abstract}

\clearpage

\begin{acknowledgements}
  I would like to express my deepest gratitude to my supervisor, Dr Jeremy Gibbons, for his
  guidance, support, comments and encouragement.
  
  I would also like to thank my family for their constant support and love.
\end{acknowledgements}

\clearpage

\pagenumbering{roman}
\pagestyle{plain}
\setcounter{tocdepth}{2}

\tableofcontents

\clearpage

\pagenumbering{arabic}
\pagestyle{myheadings}

\chapter{Introduction}
This thesis presents an extension to Behaviour Driven Development~(BDD)~\cite{North2006} to support the elicitation, communication, modelling and analysis of non-functional requirements. This extension include concepts and techniques from Goal Oriented Requirements Engineering (GORE)~\cite{Lamsweerde:2001wpba}, and specifically, it allows the specification of goals in BDD and modelling and analysis in Goal Requirements Language (GRL)~\cite{Amyot:2010kd}. This is achieved by integrating goals specifications in Gherkin~\cite{wynne2012cucumber}, a domain specific language for the representation and specification of requirements, and presenting a translator from Gherkin to GRL.

\section{Motivation}
The primary measure of success of a software system is the degree to which it meets the purpose for which it was intended~\cite{Nuseibeh:2000ub}. Shortcomings in the ways that people learn about, document, agree upon and modify such statements of intent are known causes to many of the problems in software development~\cite{Wiegers2013}. We informally refer to these statements of intent as Requirements and the engineering process to elicit, document, verify, validate and manage them as Requirements Engineering.

The importance of requirements in software engineering cannot be understated. In his essay \emph{No Silver Bullet}~\cite{Brooks1987}, Fred Brooks states that `\emph{The hardest single part of building a software system is deciding precisely what to build}'. More recently, ~\cite{Davis200505} reveals that errors introduced during requirements activities account for 40 to 50 percent of all defects found in a software product.  When arguing for the importance of requirements, \cite{Hull2011} reason that to be well understood by everybody they are generally expressed in natural language and herein lies the challenge: to capture the need or problem completely and unambiguously without resorting to specialist jargon or conventions. The authors follow by positing these needs may not be clearly defined at the start, may conflict or be constrained by factors outside their control or may be influenced by other goals which themselves change in the course of time.

Furthermore, requirements can be classified in multiple and at times conflicting ways. \cite{Glinz:2007ehba} points out that in every current classification scheme there is a distinction between requirements concerning the functionality of a system and all other, often referred to as non-functional requirements. In another paper, the same author arguments why this notion of non-functional requirements is flawed and present a new classification which is based on four facets: \emph{kind} (e.g. function, performance, or constraint), \emph{representation} (e.g. operational, quantitative or qualitative), \emph{satisfaction} (hard or soft), and \emph{role} (e.g. prescriptive or assumptive). The author points out issues with sub-classification, terminology and satisfaction level whereby some requirements are considered \emph{`soft'} in the sense that they can be weakly or strongly satisfied (e.g \emph{The system shall have a good performance}; or \emph{The System shall be secure}). 

The context just described applies to both traditional and agile software development processes. Genrically, a software process is defined as a set of activities, methods, practices, and transformations that are used to develop and maintain software and its associated products~\cite{Cugola:1998htba}. Agile software development approaches have become more popular during the last few years. Several methods have been developed with the aim to be able to deliver software faster and to ensure that the software meets customer changing needs. All these approaches share some common principles: Improved customer satisfaction, adopting requirements, frequently to changing delivering working software, and close collaboration of business people and developers~\cite{Paetsch:2003tl}.

\section{Aim and limitations of study}
The context described in the previous section justify research aimed at capturing, documenting and communicating requirements using natural language tools and techniques in a precise, complete and unambiguous way, but also with the flexibility and adaptability to allow those requirements to change and evolve through the course of time.

In this these, we do not address the open research question of requirements classification schemes and the, sometimes flawed\cite{Glinz:2007ehba}, separation of functional and non-functional requirements. Instead, we adopt the notion of goals as an objective the system under consideration should achieve. Goal formulations thus refer to intended properties to be ensured; they are optative statements as opposed to indicative ones, and bounded by the subject matter. Goals also cover different types of concerns: functional which are associated with the services to be provided, and non-functional concerns associated with quality of service --such as safety, security, accuracy, performance, and so forth.\cite{Lamsweerde:2001wpba}


\underline{\emph{TODO Rephrase:Extracted from Engineering and Managing Software Requirements}}\\
The main obstacle in decision-driven RE research resides in the desire to solve
any problem formally and rigorously. This presumes that each problem can be properly described by a formal model and is “solved” just using this model. It equates decision making with finding the optimal solution. According to Roy [51] and Schärlig [57], such thinking was also once dominant in management science and operations research. The reality was that despite enormous progress in optimization and operational research, many questions in the real world could not be answered
in a satisfactory way. In many real situations, insisting on establishing the ideal model and searching for the numerically optimal solution eventually ends in a deadlock. Such problems have been characterized as “wicked problems” by Rittel and Webber [47].

\section{Contributions}

A reinterpretation of behaviours in BDD as not just specifications of functionality of a system but as statements of goals in the spirit of GORE.

We address the vague nature of early requirements.

\section{Overview of Contents}

\begin{itemize}
\item Outline the problem
\item Explain the aim of the research
\item Set any limits to the scope of the work
\item Significance of the study
\item Provide an overview of the thesis structure
\end{itemize}

\rule{\textwidth}{1pt}

\begin{itemize}
\item Define Requirements Engineering
\item Define Software Practices and contrast Traditional vs Agile
\item Expand on Agile RE
\item Introduce BDD
\item Highlight issues with current approaches for RE and BDD
\item Expand on Functional vs Non-Functional Requirements
\item Highlight approaches for handling NFRs
\item Describe GORE and highlight benefits
\item List Research questions
\item Describe Thesis Structure 
\end{itemize}




\chapter{Background}
\label{ch:Background}
\underline{Start of chapter}
\paragraph{Link back to previous parts in particular previous chapter}
\paragraph{State the aim of the chapter}
\paragraph{Outline how you intend to achieve this aim in the form of an overview of contents}

\section{Agile Requirements Engineering}

\section{BDD}

\section{NFR Research Overview}

\section{Goal Oriented Requirements Engineering}


\paragraph*{\underline{On Non-Functional Requirements}}~\cite{Glinz:2007ehba}
\begin{itemize}
\item Definition of Requirements Engineering
\item Multiple Requirements Classification
\item Formal vs Informal specifications
\item Functional vs Non-Functional Requirements
\item Definition of N FRs
\item Issues with current definition of NFRs
\paragraph{Definition}
\begin{itemize}
\item Terminology
\item Scope
\item Misconceptions
\end{itemize}
\paragraph{Classification}
\begin{itemize}
\item Sub-Classification
\item Concepts
\end{itemize}
\paragraph{Representation}
\begin{itemize}
\item Representation-Dependant
\item Location
\item Traceability
\end{itemize}
\end{itemize}
\pagebreak

\paragraph*{\underline{A Survey of Non-Functional Requirements in Software Development Process}}~\cite{Matoussi:2008wr}
\begin{itemize}
\item Definition of Requirements Engineering
\item Formal vs Informal specifications
\item Definition of NFRs
\item Issues with current definition of NFRs

\paragraph{NFR Background Information}
\subparagraph{Requirements Analysis and Specification Level}
\begin{itemize}
\item Informal and semi-formal
\begin{itemize}
\item NFR Framework
\item Dissatisfaction driven approach
\item CEI's Quality Attribute Taxonomy
\item Use Cases based
\item PREM
\item Misuse Cases
\item KAOS
\end{itemize}
\item Formal
\begin{itemize}
\item FDAF (Formal Design Analysis Framework)
\item Specification Languages 
\item NoFun
\item Z
\end{itemize}
\end{itemize}
\subparagraph{Design Level}
\begin{itemize}
\item Use Cases and Goal Driven 
\item OONFR
\item UML based
\end{itemize}
\subparagraph{Implementation Level}
\begin{itemize}
\item Constraint and Object Oriented Programming
\item ADA
\item Exception Handling
\item Combining Rules and Object Orientation
\end{itemize}
\subparagraph{Complete approaches}
\begin{itemize}
\item FRIDA
\item NFR Reuse Process
\item BMethod
\item Tropos / Formal Tropos
\end{itemize}
\end{itemize}
\pagebreak

\paragraph*{\underline{On non-functional requirements in software engineering}}~\cite{Chung:2009vg}
\begin{itemize}
\item Functional vs Non-Functional Requirements
\item Definition of NFRs
\item{Classification Schemes}
\begin{itemize}
\item Basic and extra quality
\item Concerns
\item ISO/IEC 9126
\item NFR Frameowork based
\item Roman's Taxonomy
\item Software Quality Trees
\item FURPS
\end{itemize}
\item Issues with Classification Schemes
\item{Representations}
\subparagraph{Textual}
\begin{itemize}
\item IEEE 830
\item Volere
\item System Analysis (SADT)
\end{itemize}
\subparagraph{Trees and Lists}
\subparagraph{Use Cases and Misuse Cases}
\subparagraph{Restrictions over Scenarios}
\subparagraph{Goal oriented approaches}
\begin{itemize}
\item KAOS
\item NFR Framework
\item{i* Family}
\begin{itemize}
\item i*
\item Tropos
\item GRL
\end{itemize}
\end{itemize}
\end{itemize}

\underline{End of chapter}
\paragraph{General advice}
Start with a strong summary of the main findings of this chapter with academic references and relate it with current theory.
Relates this chapter results to earlier analysis.
End with a strong lead into next chapter.

\paragraph{Discussion or Analysis}
\begin{itemize}
\item What's important
\item What overall themes can be identified
\item What can be observed or learned
\item What limitations or shortcomings have been identified
\item Situate the chapter within the whole thesis
\end{itemize}
\paragraph{Summary}
\begin{itemize}
\item Replies to the introduction by briefly identifying the chapter's achievements and sets the scene for the next chapter
\end{itemize}
\pagebreak

\chapter{Application and reflection}

\chapter{Conclusion}

\begin{itemize}

\item Benefits of new terminology defined ~\cite{Glinz:2007ehba}  
\item Benefits of Reuse Process~\cite{Leite:2005wpba}

\end{itemize}

\bibliographystyle{apalike}
\bibliography{mydissertation}

\appendix

\end{document}

